\chapter{密立根油滴实验Python数据处理}

\section{实验原理}

动态法测量油滴电荷的方程为:

\begin{equation}
q = \frac{C}{U} 
\left(\frac{1}{t_f} + \frac{1}{t_r}\right) \left(\frac{1}{t_f}\right)^{\frac{1}{2}}
\left( \frac{1}{1 + \frac{b}{pr}} \right)^{\frac{3}{2}}
\end{equation}

平衡法测量油滴电荷的方程为:

\begin{equation}
q = \frac{C}{U} 
\left(\frac{1}{t_f}\right)^{\frac{3}{2}}
\left( \frac{1}{1 + \frac{b}{pr}} \right)^{\frac{3}{2}}
\end{equation}

其中:

\begin{equation}
C = 9\sqrt{2}\pi \left[ \frac{(\eta s)^3}{(\rho_1 - \rho_2) g} \right]^{\frac{1}{2}} d
\end{equation}

\begin{itemize}
  \item $\eta = 1.83 \times 10^{-5} $ Pa $\cdot$ s:空气粘度;
  \item $g = 9.794$ m/s$^2$:重力加速度;
  \item $\rho_1 = 979 $ kg/m$^3$:油的密度;
  \item $\rho_2 = 1.29$ kg/m$^3$:空气的密度;
  \item $P0 = 760 $mmHg:大气压强,需要换算成帕斯卡;
  \item $\rho_3 = 13.6 \times 10^3$ kg/m$^3$: 水银密度;
  \item $s = 1.6 $ mm:油滴下落距离;
  \item $d = 5$ mm:极板间距离;
  \item $b = 8.23 \times 10^{-3}$ N/m:需正常数;
\end{itemize}

\newpage
\section{测量数据}
以下为测量6颗油滴的实验数据,保存在名为:Millikan\_oil.txt的文件中请用Python读取数据):\\
============Youngs\_modulus\_data.csv============\\
\begin{longtable}{lllllllllllll}
\toprule
油滴 & U1(V) & t1(s) & U2(V) & t2(s) & U3(V) & t3(s) & U4(V) & t4(s) & U5(V) & t5(s) & U6(V) & t6(s) \\
\midrule
\endfirsthead

\toprule
\midrule
\endhead
\midrule
\multicolumn{13}{r}{{Continued on next page}} \\
\midrule
\endfoot

\bottomrule
\endlastfoot
1 & 44 & 35.40 & 33 & 24.59 &  96 & 25.57 & 248 & 24.08 & 262 & 24.01 & 136 & 6.91 \\
2 & 48 & 37.40 & 31 & 23.57 & 103 & 27.90 & 247 & 23.61 & 257 & 23.66 & 135 & 6.80 \\
3 & 44 & 37.00 & 33 & 23.55 & 102 & 25.71 & 241 & 24.88 & 255 & 23.39 & 137 & 6.75 \\
4 & 47 & 39.01 & 32 & 23.96 & 101 & 26.70 & 242 & 25.64 & 256 & 24.40 & 135 & 6.99 \\
5 & 45 & 37.96 & 32 & 24.37 & 102 & 26.68 & 244 & 26.68 & 253 & 23.81 & 136 & 6.85 \\
6 & 44 & 37.96 & 32 & 24.37 &  96 & 26.85 & 241 & 26.68 & 253 & 23.81 & 136 & 6.85
\end{longtable}

\section{数据处理}

\subsection{数据预处理}
从Millikan\_oil.txt文件中读取数据,求每颗油滴匀速下落时的平衡电压、下落时间的平均值。

\subsection{用平衡法计算电荷电量}
通过平衡法计算电荷电量。注意:计算电荷电量时,由于$q$的表达式中有油滴半径$r$,所以需要先把$r$求出来。$r$的表达式为:

\begin{equation}
r=\left[ \frac{9\eta v_f}{2g\left( \rho _1-\rho _2 \right)} \right] ^{\frac{1}{2}}\left( \frac{1}{1+\frac{b}{pr}} \right) ^{\frac{1}{2}}
\end{equation}

通过2种方法求r。

1. 当$\frac{b}{pr} \ll 1$时,

\begin{equation}
r=\left[ \frac{9\eta v_f}{2g\left( \rho _1-\rho _2 \right)} \right] ^{\frac{1}{2}}\left( \frac{1}{1+\frac{b}{pr}} \right) ^{\frac{1}{2}}\approx \left[ \frac{9\eta v_f}{2g\left( \rho _1-\rho _2 \right)} \right] ^{\frac{1}{2}}
\end{equation}

2. 令

\begin{eqnarray}
y1(r) &=& r \\
y2(r) &=& \left[ \frac{9\eta v_f}{2g\left( \rho _1-\rho _2 \right)} \right] ^{\frac{1}{2}}\left( \frac{1}{1+\frac{b}{pr}} \right) ^{\frac{1}{2}}\approx \left[ \frac{9\eta v_f}{2g\left( \rho _1-\rho _2 \right)} \right] ^{\frac{1}{2}}
\end{eqnarray}

\begin{eqnarray}
d1(r) = \left|y1(r) - y2(r)\right|
\end{eqnarray}

或者

\begin{eqnarray}
d2(r) = \left(y1(r) - y2(r)\right)^2
\end{eqnarray}

尝试画出函数$y1(r)$, $y2(r)$, $d1(r)$, $d2(r)$, 通过数值方法,在满足一定误差的条件下,求解$d1(r)$和$d2(r)$的最小值,以达到求解$r$的目的。并比较对于$d1(r)$, $d2(r)$的定义,求解方法可能会有什么不同。

\subsection{通过程序探索是否存在最小电荷即“元电荷”}

前面计算出6个油滴的带电量$q_0, q_1, \cdots, q_5$,试编程寻找是否存在最小电荷即“元电荷”。

提示:可以设定存在元电荷$e$,其范围从$1.0 \sim 2.0 \times 10^{-19} \mathrm{C}$,可以设定一个步长,比如:$de = 0.001 \times 10^{-19} \mathrm{C}$,通过循环,令$e$变化,对于给定的$e$,计算:

\begin{eqnarray}
d(e) = \frac{1}{6}\sum\limits_{i=0}^{5} \left( q_i - \left[ \frac{q_i}{e} \right] \cdot e \right)^2
\end{eqnarray}

画出曲线$d(e)$,观察有什么特点?试求出最小的$d(e)$对应的$e$,并得到相应的方差$d(e)$。比较其和元电荷$1.6 \times 10^{-19}\mathrm{C}$之间的关系。

如另有其他求解方法,请陈述之。

\subsection{是探索“提升电压”的取值合理范围}

请通过Python计算分析:针对“动态法”,若给定油滴半径为$r=1 \mathrm{\mu m}$(微米),电荷电量为$q=1.602\times 10^{-19}$C, 上升一格的时间为5s,上升8格的时间为40s,则提升电压应为多少比较合适?

\subsection{输出结果到report_Millikan.txt}
将以上整理的数据,用Python输出到report\_Millikan.txt文件中,其计算数据通过python语句进行格式化输出,按照规则保留合理的有效位数。

请将上面的图片也进行输出。