\chapter{静态法测杨氏模量实验Python数据处理}

\section{实验原理}

杨氏模量(Young's modulus,记作 $E$)描述材料在弹性形变范围内的抗拉伸或压缩能力。定义如下:

\[
E = \frac{\sigma}{\varepsilon} = \frac{F / S}{\Delta L / L} = \frac{F / \left( \frac{1}{4} \pi d^2 \right)}{ D \left( \frac{\Delta x }{2H}  \right) / L} = \frac{8 m g LH}{\pi d^2 D \Delta x}
\]

其中:

\begin{itemize}
  \item $\sigma$:应力;
  \item $\varepsilon$:应变;
  \item $F$:施加的拉力;
  \item $S$:横截面积;
  \item $L$:钢丝原始长度;
  \item $\Delta L$:伸长量;
  \item $d$:直径;
  \item $\Delta x$:标尺上光标移动的距离;
  \item $D$:光杠杆常数,即转轴到动足的距离;
  \item $H$:镜面到标尺间的距离;
\end{itemize}

\section{测量数据}
以下为测量数据,保存在名为:Youngs\_modulus\_data.csv的文件中(逗号分隔值,Comma-Separated Values,用来存储数据,可以用Excel、WPS等工具打开,也可以用Python读取或写入数据):\\
============Youngs\_modulus\_data.csv============\\
name, value\\
g,9.8,m/s\^{}2\\
L,652.0,mm\\
H,684.0,mm\\
D,38.64,mm\\
d0,0.500,mm\\
d1,0.497,mm\\
d2,0.499,mm\\
d3,0.503,mm\\
d4,0.496,mm\\
d5,0.506,mm\\
m0,1.00,kg\\
m1,1.00,kg\\
m2,1.00,kg\\
m3,1.00,kg\\
m4,1.00,kg\\
m5,1.00,kg\\
xa0,24.1,mm\\
xa1,30.3,mm\\
xa2,37.1,mm\\
xa3,43.3,mm\\
xa4,49.7,mm\\
xa5,56.1,mm\\
xb0,25.1,mm\\
xb1,31.8,mm\\
xb2,38.9,mm\\
xb3,45.1,mm\\
xb4,51.6,mm\\
xb5,57.9,mm\\
==================================================\\

\section{数据处理}

\subsection{读取数据}
请用csv模块读取文件Youngs\_modulus\_data.csv的数据。其中,部分量可以读取至列表中,例如命名为d、m、xa、xb的列表,以方便对数据进行循环遍历。

\subsection{数据预处理}
注意物理实验中要求保留有效数字的规则。尤其摄入规则采用“银行家舍入”或“IEEE 754 round half to even”(round函数)

为什么叫“银行家舍入”?

因为银行和会计处理大量的交易时,如果总是向上舍入(如传统“四舍五入”),就会导致累计误差向上偏移,最终可能形成系统性偏差。

银行家舍入通过“5看偶数”的方式,在长时间大量数据中抵消了向上或向下的偏差,使误差统计上趋于零,更公平。

请按照以下步骤计算数据相关数据,

\begin{enumerate}
  \item $\pi$ 的取值:\\
  import math \\
  pi = math.pi \\
  或者:\\
  pi = math.acos(-1) \\
  \item 计算$x_{i} = \left( x_{ai} + x_{bi} \right) / 2$
  \item 计算$dx_{ai} = (x_{i+1} - x_{i})$
  \item 计算$dx_{bi} = (x_{i+1} - x_{0})$
  \item 计算$d$的平均值$dbar = \overline{d}$,即$dbar = (1/6)\sum\limits_{i=0}^{5}d_i$,其中$\overline{d}$是数学符号,$dbar$是python变量名称。
  \item 计算$dx_{ai}$的平均值$dxa = \overline{dx_{a}}$,即$dxa = (1/6)\sum\limits_{i=0}^{5}dx_{ai}$,其中$\overline{dx_{a}}$是数学符号,$dxa$是python变量名称。
\end{enumerate}

\subsection{用增量法计算杨氏模量和其不确定度}

\begin{enumerate}
  \item 将各物理量转化为国际单位,并根据以下公式计算杨氏模量:$$E = \frac{8mgLH}{\pi \overline{d}^2 D \overline{dx_a} }$$
  \item 赋值不确定度:\\
  $uL \equiv u(L) = 0.5$ mm,\\
  $uH \equiv u(H) = 0.5$ mm,\\
  $uD \equiv u(D) = 0.02$ mm,\\
  $udi \equiv u(d_i) = 0.004$ mm\\
  $uxi  \equiv u(x_i) = 0.5$ mm
  \item 计算d的不确定度:$$ud \equiv u(\overline{d}) = \frac{u(d_i)}{\sqrt{6}} $$
  \item 计算$\overline{dx_a}$的不确定度:
  $$
  udx 
  \equiv u(\overline{dx_a})
  = 
  \sqrt{ 
        \frac{\sum\limits_{i=0}^{5} \left( dx_{ai} - \overline{dx_a} \right)^2}
        {6\times5} 
    + \left( \frac{uxi}{\sqrt{3}} \right)^2
  } $$
  \item 根据以下公式,计算杨氏模量的不确定度
  $$
  uE \equiv u(E) = E
  \sqrt{
  \left[  \frac{u(L)}{L} \right]^2 + 
  \left[  \frac{u(H)}{H} \right]^2 + 
  \left[  \frac{u(D)}{D} \right]^2 + 
  \left[ 2\frac{u(\overline{d})}{\overline{d}} \right]^2 + 
  \left[ \frac{u(\overline{dx_a})}{\overline{dx_a}} \right]^2
  }
  $$
  \item $u(E)$ 取一位有效数字,$E$ 的小数位与$u(E)$对齐,写出最后的结果:Young's Modulus = $E \pm u(E)$
\end{enumerate}

\subsection{用最小二乘法计算杨氏模量和其相关系数}

\begin{enumerate}
  \item 根据以下公式:
  $$Y_i \equiv \ dx_{bi} = \frac{8gLH}{\pi d^2 DE} m_i $$
  计算:
  $$mbar = \overline{m} = \frac{1}{6} \sum\limits_{i=0}^5 m_i$$
  $$mYbar = \overline{mY} = \frac{1}{6} \sum\limits_{i=0}^5 m_i Y_i$$
  $$Ybar = \overline{Y} = \frac{1}{6} \sum\limits_{i=0}^5 Y_i$$
  $$m2bar = \overline{m^2} = \frac{1}{6} \sum\limits_{i=0}^5 m_i^2$$
  $$Y2bar = \overline{Y^2} = \frac{1}{6} \sum\limits_{i=0}^5 Y_i^2$$
  \item 根据以下公式计算$b$:
  $$b = \frac{\overline{m}\cdot \overline{Y} - \overline{mY}}{\overline{m}^2 - \overline{m^2}}$$
   \item 根据以下公式计算杨氏模量$E$:
  $$E = \frac{8gLH}{\pi d^2 Db}$$
  \item 根据以下公式计算相关系数:
  $$gamma \equiv \gamma = 
  \frac{\overline{mY} - \overline{m} \cdot\overline{Y} }
  {\sqrt{\left[ \overline{m^2} - \overline{m}^2 \right] \left[ \overline{Y^2} - \overline{Y}^2 \right]}}$$
  \item 按照物理实验对作图的要求(《大学物理实验(第二版),牛原著,2.1.2作图法》),将以上数据点和拟合直线输出到一张图中,图片格式可以是png。
\end{enumerate}

\subsection{输出结果到report\_Young.txt}

将以上整理的数据,用Python输出到report\_Young.txt文件中,其计算数据通过python语句进行格式化输出,按照规则保留合理的有效位数。